Расчёт значения всех неизвестных токов в представленной на рисунке 1 цепи с помощью \textbf{метода узловых напряжений (МУН)}.

\subsection{Найти}
Все неизвестные токи в цепи: \(I_1, I_2, I_3, I_4, I_5 = ?\) \\
(Используя только метод узловых напряжений)

\subsection{Решение}
\begin{enumerate}
	\item Определение топологии цепи:
	      \begin{align*}
		      p^*           & = 6 \, (\text{общее количество ветвей}),                                    \\
		      p_{\text{ит}} & = 1 \, (\text{количество ветвей с источниками тока}),                       \\
		      p             & = p^* - p_{\text{ит}} = 6 - 1 = 5 \, (\text{количество неизвестных токов}), \\
		      q             & = 4 \, (\text{количество узлов}),                                           \\
		      l             & = q - 1 = 4 - 1 = 3 \, (\text{количество узловых напряжений})
	      \end{align*}

	\item Произвольно обозначим $p$ неизвестных токов и $l$ узловых напряжений, где узел с номером "0" — заземлённый. От оставшихся незаземленных узлов направляем узловые напряжения к заземленному узлу:
	      \begin{figure}[H]
	\centering
	\begin{circuitikz}[american, scale=1.4]

		\draw
		(0,0)
		to[R, l=$R_3$, i=$I_3$] (10,0)
		-- (10, -3)
		to[short, *-] (10, -3);

		\draw
		(10, -3)
		to[R, l_=$R_5$, i=$I_5$] (7.5, -3)
		to[I, l=$E_5$] (5, -3)
		to[short, *-] (5, -3)
		to[R, l=$R_2$, i=$I_2$] (2.5, -3)
		to[I, l_=$E_2$] (0, -3);

		\draw
		(0,0)
		-- (0,-3)
		to[short, *-] (0, -3)
		to[R, l=$R_1$, i_=$I_1$] (5, -6);

		\draw
		(5, -3)
		to[R, l=$R_4$, i=$I_4$] (5, -6);

		\draw (5, -6) -- (7.15, -4.7);
		\draw (7.85, -4.3) -- (10, -3);
		\draw[rotate around={-52:(7.5,-4.5)}]
		(7.5,-4.5) circle(0.4cm)
		(7.5,-4.45) edge[thick, -{Straight Barb[length=2mm]}] (7.5, -4.44)
		(7.5,-4.35) edge[thick, -{Straight Barb[length=2mm]}] (7.5, -4.34)
		node at (8.25, -4.5) {$J_6$};

		\draw (0, -3) node[draw, circle, fill=blue!10, scale=0.8] {\textcolor{blue}{1}};
		\draw (5, -3) node[draw, circle, fill=blue!10, scale=0.8] {\textcolor{blue}{2}};
		\draw (10, -3) node[draw, circle, fill=blue!10, scale=0.8] {\textcolor{blue}{3}};
		\draw (5, -6) node[draw, circle, fill=blue!10, scale=0.8] {\textcolor{blue}{0}};

		\draw[red, thick, dashed, ->]
		(0, -3.4) .. controls (0, -5) and (2.5, -6.5) .. (4.5, -6.1)
		node[midway, left, xshift=-0.2cm, red] {$U_{10}$};

		\draw[red, thick, dashed, ->] (4.7, -3.3) arc[start angle=130, end angle=230, radius=1.5] node[midway, left, red] {$U_{20}$};

		\draw[red, thick, dashed, ->]
		(10, -3.4) .. controls (10, -5) and (7.5, -6.5) .. (5.5, -6.1)
		node[midway, right, xshift=0.4cm, red] {$U_{30}$};

		\draw (5, -6.25) node[ground] {};


	\end{circuitikz}
	\caption{Схема электрической цепи с узлами, узловыми напряжениями и токами}
\end{figure}


	\item Составим систему уравнений на основе узловых напряжений. Для этого используем проводимости всех ветвей, за исключением ветвей с источниками тока. Собственные проводимости узлов — это сумма проводимостей всех ветвей, сходящихся в данный узел. В правой части уравнений находятся «узловые токи», которые включают алгебраическую сумму токов и источников энергии, действующих в ветвях узла.
	      \[
		      \begin{cases}
			      g_{11} U_{10} - g_{12} U_{20} - g_{13} U_{30} = J_{11},  \\
			      -g_{21} U_{10} + g_{22} U_{20} - g_{23} U_{30} = J_{22}, \\
			      -g_{31} U_{10} - g_{32} U_{20} + g_{33} U_{30} = J_{33}.
		      \end{cases}
	      \]

	      где \( g_{km} = g_{mk} \, (k=1\ldots l, \, m=1\ldots l, \, k \neq m) \) — общие проводимости, которые являются суммой проводимостей всех ветвей, расположенных между узлами \( k \) и \( m \) (кроме проводимости ветвей с источником тока). \( g_{kk} \) — собственные проводимости, которые представляют собой сумму проводимостей всех ветвей, сходящихся в узле \( k \) (кроме ветвей с источником тока). \( J_{kk} \) — это узловые токи, обусловленные наличием источников энергии в узлах.

	\item Запишем систему уравнений с подстановкой проводимостей и ЭДС:

	      \[
		      \begin{cases}
			      \left( \frac{1}{R_1} + \frac{1}{R_2} + \frac{1}{R_3} \right) U_{10} - \left( \frac{1}{R_2} \right) U_{20} - \left( \frac{1}{R_3} \right) U_{30} = -\frac{E_2}{R_2},                    \\
			      -\left( \frac{1}{R_2} \right) U_{10} + \left( \frac{1}{R_2} + \frac{1}{R_4} + \frac{1}{R_5} \right) U_{20} - \left( \frac{1}{R_5} \right) U_{30} = -\frac{E_2}{R_2} + \frac{E_5}{R_5}, \\
			      -\left( \frac{1}{R_3} \right) U_{10} - \left( \frac{1}{R_5} \right) U_{20} + \left( \frac{1}{R_3} + \frac{1}{R_5} \right) U_{30} = -\frac{E_5}{R_5} + J_6.
		      \end{cases}
	      \]

	\item Подставим численные значения для сопротивлений, тока и ЭДС:

	      \[
		      \begin{cases}
			      \left( \frac{1}{8} + \frac{1}{6} + \frac{1}{7} \right) U_{10} - \left( \frac{1}{6} \right) U_{20} - \left( \frac{1}{7} \right) U_{30} = -\frac{34.5}{6},                \\
			      -\left( \frac{1}{6} \right) U_{10} + \left( \frac{1}{6} + \frac{1}{5} + \frac{1}{7} \right) U_{20} - \left( \frac{1}{7} \right) U_{30} = -\frac{34.5}{6} + \frac{7}{7}, \\
			      -\left( \frac{1}{7} \right) U_{10} - \left( \frac{1}{7} \right) U_{20} + \left( \frac{1}{7} + \frac{1}{7} \right) U_{30} = \frac{7}{7} + 1.95.
		      \end{cases}
	      \]

	\item Решая данную систему, получаем значения узловых напряжений:

	      \[
		      U_{10} = 16.927 \, \text{В}, \quad U_{20} = -0.332 \, \text{В}, \quad U_{30} = 11.623 \, \text{В}.
	      \]

	\item Опреледение искомых токов через узловые напряжения:

	      \[
		      \begin{gathered}
			      I_1 = \frac{\varphi_1 - \varphi_0}{R_1} = \frac{U_{10}}{R_1} = \frac{16.927}{8} = 2.116 \, \text{А} \\
			      \\[-0.5em]
			      I_4 = \frac{\varphi_2-\varphi_0}{R_4} = \frac{U_{20}}{R_4} = \frac{-0.332}{2} = -0.166 \, \text{А} \\
		      \end{gathered}
	      \]

	      \[
		      \begin{gathered}
			      I_2 = \frac{\varphi_2 - \varphi_1 + E_2}{R_2} = \frac{(\varphi_2 -\varphi_0) - (\varphi_1 - \varphi_0) + E_2}{R_2} = \frac{(U_{20}-U_{10}+E_2)}{R_2} = \\
			      = \frac{-0.332-16.927+34.5}{6} = 2.874 \, \text{А} \\
			      I_3 = \frac{\varphi_1 - \varphi_3}{R_3} = \frac{(\varphi_1 -\varphi_0) - (\varphi_3 - \varphi_0)}{R_3} = \frac{U_{10}-U_{30}}{R_3} = \\
			      = \frac{16.927-11.623}{7} = 0.758 \, \text{А} \\
			      I_5 = \frac{\varphi_3-\varphi_2+E_5}{R_5} = \frac{(\varphi_3-\varphi_0)-(\varphi_2-\varphi_0)+E_5}{R_5} = \frac{U_{30}-U_{20}+E_5}{R_5} = \\
			      = \frac{11.623+0.332+7}{7} = 2.708 \, \text{А} \\
		      \end{gathered}
	      \]
\end{enumerate}


\subsection{Ответ}
Рассчитанные значения неизвестных токов в цепи:

\[
	I_1 = 2.116 \, \text{А}, \quad
	I_2 = 2.874 \, \text{А}, \quad
	I_3 = 0.758 \, \text{А}, \quad
	I_4 = -0.166 \, \text{А}, \quad
	I_5 = 2.708 \, \text{А}.
\]

Все токи найдены с использованием Метода узловых напряжений и полностью совпадают со значениями, полученными с помощью I и II законов Кирхгофа, что подтверждает правильность расчётов \textit{первой} и \textit{второй} частей.
