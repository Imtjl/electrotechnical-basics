Расчёт значения всех неизвестных токов в представленной на рисунке 1 цепи с помощью Законов Кирхгофа.

\subsection{Найти}
Все неизвестные токи в цепи: \(I_1, I_2, I_3, I_4, I_5 = ?\) \\
(Используя только I и II законы Кирхгофа)

\subsection{Решение}
\begin{enumerate}
	\item \textbf{Постоянная времени } $\tau$ \textbf{в RC-цепи} рассчитывается по формуле:

	      \[
		      \tau = R \cdot C
	      \]

	      где $R = 40 \, \Omega$ и $C = 300 \, \mu F$.

	      Подставим значения:

	      \[
		      \tau = 40 \, \Omega \cdot 300 \cdot 10^{-6} \, F \cdot 10^{-3} = 12 \, \text{мс}
	      \]

	\item Расчётные значения тока и напряжения на конденсаторе после коммутации, а также установившиеся значения напряжения на конденсаторе и тока в цепи для RC-цепи рассчитываются по следующим формулам:

	      \[
		      \lim_{t \to 0^+} U_C(t) = \lim_{t \to 0^-} E(t) = -2 \, \text{В}
	      \]

	      \[
		      \lim_{t \to 0^+} I(t) = \frac{E + U_C}{R} = \frac{2\, \text{В} + 2\, \text{В}}{40 \, \Omega} = 100 \, \text{мА}
	      \]

	      \[
		      \lim_{t \to \infty} I(t) = \lim_{t \to 0^-} I(t) = 0 \, \text{мА}
	      \]

	      \[
		      \lim_{t \to \infty} U_C(t) = \lim_{t \to 0^+} E(t) = 2 \, \text{В}
	      \]

	\item Значение времени переходного процесса \( t_{0.5} \) определяется как время, за которое ток достигает \textit{половины} своего амплитудного значения. Постоянная времени \( \tau \) определяется как:

	      \[
		      \tau = \frac{t_{0.5}}{\ln 2}
	      \]

	      \textbf{Вычисление постоянной времени } \( \tau \):

	      \[
		      \tau = \frac{t_{0.5_I}}{\ln 2} = \frac{8.418554 \, \text{мс}}{0.69314718} = 12.144 \, \text{мс}
	      \]

\end{enumerate}

Постоянная времени \( \tau \) была определена экспериментально и составляет \( 12.144 \, \text{мс} \). Это значение будет использовано для расчёта соответствующих токов и напряжений в цепи.


\subsection{Ответ}
Рассчитанные значения неизвестных токов в цепи:

\[
I_1 = 2.116 \, \text{А}, \quad
I_2 = 2.874 \, \text{А}, \quad
I_3 = 0.758 \, \text{А}, \quad
I_4 = -0.166 \, \text{А}, \quad
I_5 = 2.708 \, \text{А}.
\]

Все токи найдены с использованием I и II законов Кирхгофа.
