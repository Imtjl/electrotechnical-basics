\begin{enumerate}
	\item Коэффициент затухания:

	      \[
		      \delta = \frac{R}{2L} = \frac{160}{2 \cdot 0,48} = \frac{500}{3} \, \text{с}^{-1}
	      \]

	\item Резонансная частота:

	      \[
		      \omega_o = \sqrt{\frac{1}{LC}} = \sqrt{\frac{1}{0,48 \cdot 300 \cdot 10^{-6}}} = \frac{250}{3} \, \text{с}^{-1}
	      \]

	\item Корни характеристического уравнения:

	      \[
		      \begin{gathered}
			      s_{1,2} = -\delta \pm \sqrt{\delta^2 - w_o^2} \\
			      s_1 = -\delta - \sqrt{\delta^2 - w_o^2} = -\frac{500}{3} - \sqrt{\frac{500^2}{3^2} - \frac{250^2}{3^2}} \approx -311,004 \\
			      s_2 = -\delta + \sqrt{\delta^2 - w_o^2} = -\frac{500}{3} + \sqrt{\frac{500^2}{3^2} - \frac{250^2}{3^2}} \approx -22,329 \\
		      \end{gathered}
	      \]

	\item Общее изменение напряжения при коммутации:

	      \[
		      E_\Sigma = (|E(0-)| + |E(0+)|) \cdot sign(E(0+)) = (2 + 2) \cdot 1 = 4 \, \text{В}
	      \]

	\item Напряжение на конденсаторе после коммутации:

	      \[
		      \begin{gathered}
			      U_C(t) = E(0+) - \frac{E_\Sigma}{s_1 - s_2}(s_1 e^{s_2 t} - s_2 e^{s_1 t}) \\
			      \lim_{t \to 0^+} U_C(t) = 2 - \frac{4}{-311,004+22,329}(-311,004 \cdot e^{-22,329 \cdot 0} + 22,329 \cdot e^{-311,004 \cdot 0}) = \\
			      = 2 - \frac{4}{\cancel{-311,004+22,329}}(\cancel{-311,004 \cdot 1 + 22,329 \cdot 1}) = -2 \, \text{В}
		      \end{gathered}
	      \]

	\item Напряжение на индуктивном элементе после коммутации:

	      \[
		      \begin{gathered}
			      U_L(t) = \frac{E_\Sigma}{s_1 - s_2}(s_1 e^{s_2 t} - s_2 e^{s_1 t}) \\
			      \lim_{t \to 0^+} U_L(t) = \frac{4}{-311,004+22,329}(-311,004 \cdot e^{-22,329 \cdot 0} + 22,329 \cdot e^{-311,004 \cdot 0}) = \\
			      = \frac{4}{\cancel{-311,004+22,329}}(\cancel{-311,004 \cdot 1 + 22,329 \cdot 1}) = 4 \, \text{В}
		      \end{gathered}
	      \]

\end{enumerate}





















Значение времени переходного процесса \( t_{0.5} \) определяется как время, за которое напряжение на конденсаторе достигает половины своего установившегося значения. Постоянная времени \( \tau \) определяется как:

\[
	\tau = \frac{t_{0.5}}{\ln 2}
\]

\textbf{Вычисление постоянной времени \( \tau \):}

\[
	\tau = \frac{t_{0.5}}{\ln 2} = \frac{8.41833 \, \text{мс}}{0.69314718} = 12.144 \, \text{мс}
\]
