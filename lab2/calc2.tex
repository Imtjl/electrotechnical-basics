1. \, \text{Постоянная времени } \tau:

\[
	\tau = \frac{L}{R} = \frac{0.48}{40} = 0.012 \, \text{с} = 12 \, \text{мс}
\]

2. \, \text{Ток в момент коммутации } I (0+):

\[
	I(0+) = I(0-) = \frac{E(0-)}{R} = \frac{-2}{40} = -0.05 \, \text{А} = - 50 \, \text{мА}
\]

3. \, \text{Напряжение на катушке } $U_L(0+)$:

\[
	U_L(0+) = E(0+) - I(0-) \cdot R = 2 - (-0.05 \cdot 40) = 2 + 2 = 4 \, \text{В}
\]

4. \, \text{Установившийся ток } $I(\infty)$:

\[
	I(\infty) = \frac{E}{R} = \frac{2}{40} = 0.05 \, \text{А} = 50 \, \text{мА}
\]

5. \, \text{Напряжение на катушке в установившемся режиме } $U_L(\infty)$:

\[
	U_L(\infty) = I(\infty) \cdot R_k = 50 \cdot 0 = 0 \, \text{В}
\]

Значение времени переходного процесса \( t_{0.5} \) определяется как время, за которое напряжение достигает половины своего амплитудного значения. Постоянная времени \( \tau \) определяется как:

\[
	\tau = \frac{t_{0.5}}{\ln 2}
\]

\textbf{Вычисление постоянной времени \( \tau \):}

\[
	\tau = \frac{t_{0.5_{U_L}}}{\ln 2} = \frac{8.418331 \, \text{мс}}{0.69314718} = 12.143 \, \text{мс}
\]


Постоянная времени \( \tau \) была определена экспериментально и составляет
\( 12.144 \, \text{мс} \). Это значение будет использовано для расчёта соответствующих токов и напряжений в цепи.
