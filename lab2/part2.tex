\subsection{Апериодический процесс}

\subsubsection{Схема исследуемой цепи}
На рисунке 1.1 представлена схема замещения источника электрической энергии постоянного тока и нагрузки, созданная в приложении LTspice.

% \begin{figure}[H]
% 	\centering
% 	\includegraphics[width=0.6\textwidth]{rcl-schema.png} % Make sure the path to the image is correct
% 	\caption{Схема замещения источника электрической энергии в LTspice.}
% \end{figure}

\subsubsection{Расчётные формулы и расчёты}

Значение времени переходного процесса \( t_{0.5} \) определяется как время, за которое напряжение на конденсаторе достигает половины своего установившегося значения. Постоянная времени \( \tau \) определяется как:

\[
\tau = \frac{t_{0.5}}{\ln 2}
\]

\textbf{Вычисление постоянной времени \( \tau \):}

\[
\tau = \frac{t_{0.5}}{\ln 2} = \frac{8.41833 \, \text{мс}}{0.69314718} = 12.144 \, \text{мс}
\]

\subsubsection{Графики переходных процессов}

\subsubsection{Таблица результатов 4.4}

\subsection{Колебательный процесс}

\subsubsection{Схема исследуемой цепи}
На рисунке 1.1 представлена схема замещения источника электрической энергии постоянного тока и нагрузки, созданная в приложении LTspice.

% \begin{figure}[H]
% 	\centering
% 	\includegraphics[width=0.6\textwidth]{rcl-schema.png} % Make sure the path to the image is correct
% 	\caption{Схема замещения источника электрической энергии в LTspice.}
% \end{figure}

\subsubsection{Расчётные формулы и расчёты}

\subsubsection{Графики переходных процессов}

\subsubsection{Таблица результатов 4.5}

\subsection{Выводы по второй части}
