\newpage
\section{Пример расчёта для одной строки таблицы}

Для расчёта параметров используем следующие формулы:

\begin{itemize}
	\item Ток через нагрузку:
	      \[
		      I_n = \frac{U_n}{R_n}
	      \]

	\item Мощность, рассеиваемая на нагрузке:
	      \[
		      P_n = \frac{U_n^2}{R_n}
	      \]

	\item Коэффициент полезного действия:
	      \[
		      \eta_n = \frac{R_n}{R_n + r}
	      \]

	\item Внутреннее сопротивление источника:
	      \[
		      r_k = \frac{U_k - U_{k+1}}{I_{k+1} - I_k}
	      \]
\end{itemize}

Рассчитаем значения для строки \(n = 2\):

\begin{align*}
	I_2    & = \frac{U_2}{R_2} = \frac{10.692}{5400} = 1.98 \, \text{мА},                           \\
	P_2    & = \frac{U_2^2}{R_2} = \frac{10.692^2}{5400} \approx 0.021 \, \text{Вт},                \\
	\eta_2 & = \frac{R_2}{R_2 + r} = \frac{5400}{5400 + 600} = 0.9,                                 \\
	r_2    & = \frac{U_2 - U_3}{I_3 - I_2} = \frac{10.692 - 9.504}{3.96 - 1.98} = 600 \, \text{Ом}.
\end{align*}

Таким образом, для строки \(n = 2\) были рассчитаны следующие значения:
\[
	I_2 = 1.98 \, \text{мА}, \quad P_2 = 0.021 \, \text{Вт}, \quad \eta_2 = 0.9, \quad r_2 = 600 \, \text{Ом}.
\]
