\newpage
\section{Расчётная внешняя характеристика источника}

На рисунке \ref{fig:characteristic} представлена расчётная и экспериментальная внешняя характеристика источника.
Расчётная характеристика изображена в виде синей линии,
которая соединяет точки \((0, E = 12 \, \text{В})\) и \((I_{sc} = 20 \, \text{мА}, 0)\).
Эта линия отражает теоретическую зависимость напряжения на нагрузке \(U_n\) от тока \(I_n\),
поступающего от источника, при идеальных условиях.

Экспериментальные точки, отмеченные на графике красными квадратами,
соответствуют измеренным значениям напряжения \(U_n\) для разных токов \(I_n\),
согласно данным из таблицы 1.1. Эти точки показывают реальные данные,
полученные при изменении сопротивления нагрузки, и их отклонения от расчётной линии
могут свидетельствовать о наличии потерь или неточностей в измерениях и/или вычислениях.

\begin{figure}[h]
	\centering
	\begin{tikzpicture}
		\begin{axis}[
				width=17cm, height=14cm, % Размер графика
				xlabel={$I_n$ [мА]}, % Ось X
				ylabel={$U_n$ [В]},  % Ось Y
				axis lines=middle,
				grid=major,          % Основная сетка
				xmin=0, xmax=22,     % Диапазон по оси X
				ymin=0, ymax=13,     % Диапазон по оси Y
				domain=0:20,         % Область построения
				thick,               % Толщина линии
				legend style={at={(0.95,0.95)}, anchor=north east}, % Легенда
				label style={font=\small},
				tick label style={font=\small},
				xtick={0, 2.5, 5, 7.5, 10, 12.5, 15, 17.5, 20}, % Подписи по оси X
				ytick={0, 1, 2, 3, 4, 5, 6, 7, 8, 9, 10, 11, 12}, % Подписи по оси Y
			]

			% Линия расчетной характеристики
			\addplot[color=blue, thick] coordinates {
					(0,12) (20,0)
				};
			\addlegendentry{Расчётная характеристика}

			% Экспериментальные точки
			\addplot[color=red, only marks, mark=square*] coordinates {
					(0,12)
					(2.00, 10.8)
					(4.00, 9.6)
					(6.00, 8.4)
					(8, 7.2)
					(10, 6)
					(12.00, 4.8)
					(14.004, 3.599)
					(16, 2.4)
					(17.985, 1.2)
					(20, 0)
				};
			\addlegendentry{Экспериментальные точки}

		\end{axis}
	\end{tikzpicture}
	\caption{График расчётной и экспериментальной внешней характеристики источника}
	\label{fig:characteristic}
\end{figure}
